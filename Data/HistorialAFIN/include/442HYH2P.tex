% a2 {{{
\begin{enumerate}
\item {\bf 2019}: Comella, A; Casas- Baroy, J.C.; Marc-Armengual, J.M.; Serra, A.; Obradors, N.; Busquets, E.; Galbany Estragués, P.; Pujol, R.  Percepción de las dificultades académicas y las circunstancias que pueden condicionar el rendimiento en los estudiantes de primer año en los grados de ciencias de la salud.. {\it FEM: Revista de la Fundación de Educación Médica}, 22, 24-25.  \filbreak
\item {\bf 2019}: De Lorenzi, M.  Nuevos caminos entre viejos campos – Pluriparentalidades en tránsito. {\it Revista de Derecho de Familia},  \filbreak
\item {\bf 2019}: De Lorenzi, M.  Tomando vuelo… Perfiles actuales de las uniones convivenciales. Revista de Derecho de Familia. {\it Revista de Derecho de Familia}, 92.  \filbreak
\item {\bf 2019}: De Lorenzi, M; Gallego Molinero, A.; Fernández, P.  Adopción y Derechos. El acceso a los orígenes en Argentina, Chile y España. {\it Revista de Ciencias Sociales "América Latina Hoy"}, 83, 7-23.  \filbreak
\item {\bf 2019}: De Lorenzi, M; Lloveras, N.  Cambio de residencia de los hijos menores de edad. Preguntas y respuestas. {\it DFyP}, 17.  \filbreak
\item {\bf 2019}: De Lorenzi, M; Varas, M.G.  Familias y Pluriparentalidades. ¿Un puzzle por armar?. {\it Revista de Derecho de Familia},  \filbreak
\item {\bf 2019}: Marc-Amengual, Jaume-Miquel; Comella, A; Serra, A.; Casas- Baroy, J.C.; Riera, B.; Galbany Estragués, ; Pujol, R.  Burnout académico al empezar la Universidad en grados de ciencias de la salud. {\it FEM: Revista de la Fundación de Educación Médica}, 22, 23-24.  \filbreak
\item {\bf 2020}: Platero, R.L.; López-Sáez, M.A.  “Perder la propia identidad”. La adolescencia LGTBQA+ frente a la pandemia por COVID-19 y las medidas del estado de alarma en España. {\it Sociedad e Infancias}, 4, 95-98.  \\ DOI: https://doi.org/10.5209/soci.69358%20ISSN%202531-0720. \filbreak
\item {\bf 2020}: Sanmiquel-Molinero, Laura.  Los Estudios de la Dis/capacidad: una propuesta no individualizante para interrogar críticamente la producción del cuerpo-sujeto discapacitado. {\it Papeles del CEIC}, 2020/2(Paper 231), Papel 231.  \\ DOI: 10.1387/pceic.20974. \filbreak
\item {\bf 2021}: Blanco Fuente, Irene; López-Sáez, Miguel Ángel; Platero, R. Lucas.  Mascarillas, cams apagadas, plumas y voces. {\it Concreta}, 17, 140-147.  \filbreak
\item {\bf 2021}: Comella, Agustí; Casas-Baroy, Joan-Carles; Comella-Company, Anna; Galbany-Estragués, Paola; Pujol, Ramon; Marc-Amengual, Jaume-Miquel.  Burnout y rendimiento académico: efecto de la combinación de la actividad laboral remunerada e iniciar los estudios de grado universitario (Burnout and academic performance: effect of the combination of remunerated jobs and starting university degree stu. {\it Retos}, 41, 844-853.  \\ DOI: 10.47197/retos.v41i0.85971. \filbreak
\item {\bf 2022}: Malgosa, Estel; Alvarez, Bruna; Marre, Diana.  Sexualitat i infància a Catalunya, Espanya: significacions gobernades. {\it Perifèria, revista de recerca i formació en antropologia}, 27(2), 4-29.  \\ DOI: https://doi.org/10.5565/rev/periferia.894. \filbreak
\item {\bf 2022}: McIntyre, Lynne; Alvarez, Bruna; Marre, Diana.  “I want to bury it. Will you join me?”: The use of ritual in prenatal loss among women in Catalonia, Spain in the early 21st century. {\it Religions}, 13(4), 336.  \\ DOI: https://doi.org/10.3390/rel13040336. \filbreak
\item {\bf 2023}: Hernández Villasol, Raúl; Marre, Diana; Carrasco Pons, Sílvia.  Moldear la pertenencia Social. Adolescentes jugando en terapia. {\it Perifèria. Revista de Recerca i  Formació en Antropologia}, 28(1), 4-26.  \\ DOI: https://doi.org/10.5565/rev/periferia.910. \filbreak
\item {\bf 2023}: Poveda, .  "Entonces me empecé a sentir sola”: La revelación accidental del origen adoptivo como coda autobiográfica en personas adultas adoptadas en Chile. {\it Papers infancia\_c}, 27, 1-26.  \filbreak
\item {\bf 2023}: Sanmiquel-Molinero, Laura.  Reseña de 'El cuerpo deseado: la conversación pendiente entre feminismo y anticapacitismo'. {\it Revista Española de Discapacidad}, 11(1), 267-270.  \filbreak
\item {\bf 2024}: Marre, Diana.  Reconèixer i integrar la diversitat familiar. {\it Barcelona Metrópolis}, 129, 16-20.  \filbreak
\end{enumerate} 
% }}}
