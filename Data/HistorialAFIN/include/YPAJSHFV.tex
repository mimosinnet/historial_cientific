% a4 {{{
\begin{enumerate}
\item {\bf 2019}: Fonseca, Claudia; Marre, Diana. Adoção transnacional e humanitarismo: Resgate ou Rapto?. {\bf In} Gabriel Gatti (Ed.): {\it Narrativas, figuras y soportes de la transnacionalización del sufrimiento. De los desaparecidos (locales) a las víctimas (globales} (pp. ). Leioa, Bizkaia, Spain: Universidad del País Vasco. {\bf }\filbreak
\item {\bf 2019}: Insausti, Santiago Joaquin. Las epopeyas de la memoria gay. {\bf In}  (Ed.): {\it Los mil pequeños sexos. Intervenciones críticas sobre políticas de género y sexualidades} (pp. ). Buenos Aires, Argentina: Universidad Nacional de Tres de Febrero. {\bf }\filbreak
\item {\bf 2019}: Ramírez-March, Álvaro; Andrés, Jaime; Montenegro, Marisela. Research as Social Engagement: Learning From Two Situated Experiences in Zaragoza and Barcelona. {\bf In} Marjan Ivković, Srđan Prodanović (Ed.): {\it Engaging (for) Social Change: Towards New Forms of Collective Action} (pp. 259-282). Belgrade: Institute for Philosophy and Social Theory. {\bf }\filbreak
\item {\bf 2020}: Alvarez, Bruna. Co-Parents Who Share Family Work: Feminism, Co-responsibility and “Mother Knows Best” in Spanish Heterosexual Couples.. {\bf In} N Willey, D Friedman (Ed.): {\it Feminist Fathering / Fathering Feminists} (pp. 219-238). Toronto: Demeter Press. {\bf }\filbreak
\item {\bf 2020}: Alvarez, Bruna; Malgosa, Estel. La diversidad y la sexualidad en las familias: la perspectiva antropológica para la transformación social.. {\bf In} Bruna Alvarez, J.C. García-Reyes, M.D. García-Reyes, I Malaver, E Malgosa, M.L Paituví (Ed.): {\it Las ciencias sociales en la práctica} (pp. 20-38). Barcelona: Universitat Oberta de Catalunya. {\bf }\filbreak
\item {\bf 2021}: Leinaweaver, Jessaca B.; Marre, Diana. Adoption and Fostering. {\bf In} S Han, C Tomori (Ed.): {\it The Routledge Handbook of Anthropology and Reproduction.} (pp. 618-630). Abingdon, UK and Philadelphia: PA: Routledge. {\bf }\filbreak
\item {\bf 2021}: San Román, Beatriz. Waiting Too Long to Mother: Involuntary Childlessness and Assisted Reproduction in Contemporary Spain. {\bf In} Marcia Inhorn, Nancy J. Smith-Hefner (Ed.): {\it Waithood: Gender, Education, and Delays in Marriage and Childbearing} (pp. 339-361). Oxford: Berghahn Books. {\bf }\filbreak
\item {\bf 2021}: Vergara del Solar, Ana; Sepulveda, Mauricio; Salvo Agoglia, Irene. Understanding and Caring for Parents: Moral Reflexivity in the Discourse of Chilean Children. {\bf In}  (Ed.): {\it Bringing Children Back into the Family: Relationality, Connectedness and Home} (pp. ). FALTA COMPLETAR: FALTA COMPLETAR. {\bf }\filbreak
\item {\bf 2022}: Alvarez, Bruna; Malgosa, Estel; Marre, Diana. Ethnography on sensitive topics: Children's sexuality education in Spain. {\bf In} Jenna Pandelli, Hugo Gaggiotti, Neil Sutherland (Ed.): {\it Organizational Ethnography: An Experiential and Practical Guide} (pp. 126-140). FALTA COMPLETAR: Routledge. {\bf }\filbreak
\item {\bf 2022}: Desy, Alexandra; Marre, Diana. Reproductive exclusion: French clients undergoing cross-border reproductive care in Barcelona. {\bf In} Corinna Guerzoni, Claudia Mattalucci (Ed.): {\it Body Politics and Reproductive Governances: “Flesh”, Technologies and Knowledge} (pp. 163-177). Bingley: Emerald Group. {\bf }\filbreak
\item {\bf 2022}: Insausti, Santiago Joaquin. Lentejuelas, calabozos y políticas: una historia de las luchas travestis en la Ciudad de Buenos Aires (1985-2004). {\bf In}  (Ed.): {\it Nueva historia de las mujeres en Argentina} (pp. ). Buenos Aires, Argentina: Prometeo. {\bf }\filbreak
\item {\bf 2022}: Lewis-Jackson, Sasha; Iob, Eleonora; Giunchiglia, Valentina; Cabral, Jose Roberto; Romeu-Labayen, Maria; Cooper, Silvie; Makamu, Rirhandzu; Dorasamy, Cassandra; Ncube, Matthew; Chasara, Romeo; Bautista-González, Elysse; Mendiola, Paulina Pérez-Duarte; Huapaya, Victoria Cavero; Davies, Megan; Rutz, Florian; Staudacher, Sandra. Policies and Politics: An Analysis of Public Policies Aimed at the Reorganisation of Healthcare Delivery During the COVID-19 Pandemic. {\bf In} Cecilia Vindrola-Padros, Ginger A. Johnson (Ed.): {\it Caring on the Frontline during COVID-19: Contributions from Rapid Qualitative Research} (pp. 39-64). Singapore: Springer. {\bf Impact: DOI: 10.1007/978-981-16-6486-1\_3.}\filbreak
\item {\bf 2022}: Romeu-Labayen, Maria; Alvarez, Bruna; Block, Ellen; Cabral, José Roberto; Diallo, Marème; Djellouli, Nehla; Galbany-Estragués, Paola; Hoernke, Katarina; Marre, Diana; Moglia, Brenda; Pasarin, Lorena; Remorini, Carolina; Rivera, Priscila; Subías, Martí; Sy, Anahi; Tort-Nasarre, Glòria; Vázquez-Segura, Eva; Yan, Shirley. Protecting and Feeling Protected: HCWs' Experiences with Personal Protective Equipment During the COVID-19 Pandemic (PPE). {\bf In} Cecilia Vindrola-Padrós, Ginger Johnson (Ed.): {\it Caring on the Frontline during COVID-19} (pp. 65-90). FALTA COMPLETAR: Springer Singapore. {\bf Impact: DOI: 10.1007/978-981-16-6486-1\_4.}\filbreak
\item {\bf 2022}: Romeu-Labayen, Maria; Alvarez, Bruna; Block, Ellen; Cabral, José Roberto; Diallo, Marème; Djellouli, Nehla; Galbany-Estragués, Paola; Hoernke, Katarina; Marre, Diana; Moglia, Brenda; Pasarin, Lorena; Remorini, Carolina; Rivera, Priscila; Subías, Martí; Sy, Anahi; Tort-Nasarre, Glòria; Vázquez-Segura, Eva; Yan, Shirley. Protecting and Feeling Protected: HCWs’ Experiences with Personal Protective Equipment During the COVID-19 Pandemic (PPE). {\bf In} Cecilia Vindrola-Padros, Ginger A. Johnson (Ed.): {\it Caring on the Frontline during COVID-19: Contributions from Rapid Qualitative Research} (pp. 65-90). Singapore: Springer. {\bf Impact: DOI: 10.1007/978-981-16-6486-1\_4.}\filbreak
\item {\bf 2023}: Insausti, Santiago Joaquin. Gloria Meneses, memorias de la travesti mas vieja de america.. {\bf In}  (Ed.): {\it Memoria a la intemperie: (auto)biografías trans hispánicas} (pp. ). Lleida, España: Universidad de Lleida. {\bf }\filbreak
\item {\bf 2023}: Marre, Diana; Clemente, Chandra. Suprimir la Identidad para “reparar” el (mal) Origen. {\bf In} Mariana (ed.) De Lorenzi (Ed.): {\it Derecho a conocer los orígenes de niñas, niños y adolescentes. Una mirada multidisciplinar.} (pp. 99-118). Buenos Aires: Rubinzal Culzoni. {\bf }\filbreak
\item {\bf 2023}: Marre, Diana; Leinaweaver, Jessaca B. Disappearance via Adoption: On Missing Children in Spain (1936–96). {\bf In} Laura Huttunen, Gerhild Perl (Ed.): {\it An Anthropology of Dissappearance. Politics, Intimacies and the Troubling Question of Knowing} (pp. 119-141). FALTA COMPLETAR: Berghahn Books. {\bf }\filbreak
\item {\bf 2023}: Salvo Agoglia, Irene. La comunicación de los orígenes: de la “revelación” de los secretos a la apertura comunicativa. {\bf In}  (Ed.): {\it Derecho a conocer los orígenes de niñas, niños y adolescentes. Reflexiones desde una mirada interdisciplinar.} (pp. ). FALTA COMPLETAR: Rubinzal – Culzoni Editores.. {\bf }\filbreak
\item {\bf 2023}: Salvo Agoglia, Irene. La relevancia de los orígenes en la construcción de la identidad: algunas evidencias interdisciplinarias. {\bf In}  (Ed.): {\it Derecho a conocer los orígenes de niñas, niños y adolescentes. Reflexiones desde una mirada interdisciplinar} (pp. ). FALTA COMPLETAR: FALTA COMPLETAR. {\bf }\filbreak
\item {\bf 2024}: Insausti, Santiago Joaquin. Sexuality in a distant Metropoly. Buenos Aires, 1880-2020. {\bf In}  (Ed.): {\it Cambridge World History of Sexualities.} (pp. ). Cambridge: Cambridge. {\bf }\filbreak
\item {\bf 2024}: Insausti, Santiago. Political, Ethical, and Methodological Concerns in the Digital Humanities: Personal Reflections on the Robert Roth Papers and Transnational Queer Networks.. {\bf In}  (Ed.): {\it Transnational LGBTQ+ Networks and Activism in Europe and the Americas, 1940s–2000s: Collaborations, Interventions, Language, and Desire .} (pp. ). London: Bloomsbury. {\bf }\filbreak
\item {\bf 2024}: Insausti, Santiago. Maricas trabajando: un acercamiento a la historia del trabajo queer (Buenos Aires, 1980-1880). {\bf In}  (Ed.): {\it Perseguidos y perseguidores. Estudios sobre género, trabajo y represión en la historia argentina reciente.} (pp. ). Buenos Aires: Prometeo. {\bf }\filbreak
\item {\bf 2024}: Insausti, Santiago. Las condiciones historicas de la ley de matrimonio igualitario en Argentina. {\bf In}  (Ed.): {\it TRAYECTORIAS DE LOS ESTUDIOS FEMINISTAS DEL GÉNERO: una antología desde y para América Latina.} (pp. ). México: Universidad Iberoamericana Puebla. {\bf }\filbreak
\item {\bf 2024}: Sanmiquel-Molinero, Laura; García-Santesmases Fernández, Andrea. Cuidados (in)sostenibles: un análisis feminista y anti-capacitista del trabajo social con personas en situación de dependencia. {\bf In} Eva María Rubio-Guzmán, Jesús M. Pérez-Viejo, Francisco Javier García-Castilla (Ed.): {\it La interseccionalidad. Un enfoque clave para el trabajo social} (pp. 155-168). Madrid: Dykinson. {\bf Impact: SPI2022=3.}\filbreak
\end{enumerate} 
% }}}
