% a5 {{{
\begin{enumerate}
\item {\bf 2019}: Alvarez, Bruna. {\it Itinerarios reproductivos: deseos, derechos y estratificación social.}. XI Congreso Internacional AFIN: Hacia una (in)justicia reproductiva: Movilidades, Tecnologías, Trabajos, Decisiones, , Granada (España).\filbreak
\item {\bf 2019}: Alvarez, Bruna. {\it Las inmovilidades del anonimato en la recepció de gametos donados: receptoras de gametos en España.}. V Congreso Internacional AIBR: pensar culturas, cambiar mundos., , Madrid (España).\filbreak
\item {\bf 2019}: Alvarez, Bruna. {\it No-madres: mercado laboral y reproducción.}. IV Col.loqui MARC: sexe, gènere i salut., , Tarragona (España).\filbreak
\item {\bf 2019}: Clemente-Martínez, . {\it “In first person”}. Participation in colloquium. 3rd Galician Conference on Adoption and Fostering. MANAIA, , Pontevedra (Spain).\filbreak
\item {\bf 2019}: Clemente-Martínez, Chandra Kala. {\it Kinship in Spanish adoptive families and the search for origins}. Seminar for “Basic Principles of Social Anthropology”. Bachelor of Gender Sociocultural Studies. Autonomous University of Barcelona., , Barcelona (Spain).\filbreak
\item {\bf 2019}: Clemente-Martínez, Chandra Kala. {\it Personal experience of adoptive identity: Differences and belongings}. Seminar for “Basic Principles of Social Anthropology”. Bachelor of Gender Sociocultural Studies. Autonomous University of Barcelona, , Barcelona (Spain).\filbreak
\item {\bf 2019}: Clemente-Martínez, Chandra Kala. {\it Transnational adoptions and journeys of return to origins}. V International Congress of Anthropology AIBR, Thinking Cultures, Changing Worlds, , Madrid (España).\filbreak
\item {\bf 2019}: Clemente-Martínez, Chandra Kala. {\it Què significa tornar i viure amb la cultura d’origen? [What does it mean to return and live with the culture of origin?]}. Tertúlia ADDIF. Associació en Defensa del Dret de la Infància a la Família, , Barcelona (Spain).\filbreak
\item {\bf 2019}: Clemente-Martínez, Chandra Kala. {\it La cerca dels orígens. Quan de sobte hi és, i no només al teu cap, la familia biològica [Search for Origins: When suddenly birth family is there, and not only in your mind]}. Tertúlia ADDIF. Associació en Defensa del Dret de la Infància a la Família, , Barcelona (Spain).\filbreak
\item {\bf 2019}: Clemente-Martínez, Chandra Kala. {\it Back to the origins: Searches and reunions between adoptive families and families of origin}. Experiences in the search for origins. Adoption: training sessions. UME Alaia Guipuskoa, , San Sebastián (Spain).\filbreak
\item {\bf 2019}: Clemente-Martínez, Chandra Kala. {\it “We Only Gave Him Birth…”: Birth Families Searching their Children}. Conference. Towards Reproductive (In)Justice: Mobilities, Technologies, Labourings \& Decisions. AFIN, , Granada (Spain).\filbreak
\item {\bf 2019}: Clemente-Martínez, Chandra Kala; Guerra, Diana. {\it Narrating and searching for 'origins' in gamete donation and adoption}. Conference Being diverse families today, , Centro Carmen Jiménez. Granada (Spain).\filbreak
\item {\bf 2019}: Clemente-Martínez, Chandra Kala; Marre, Diana. {\it Kinship in Spanish Adoptive Families and Searches for Origins}. Contested Kinship: Towards a Redefinition of Human Relations, , Georg-August-Universität Gottingen. Gottingen (Germany).\filbreak
\item {\bf 2019}: Malgosa, Estel. {\it Percepciones infantiles sobre reproducción y sexualidad}. XI International AFIN Conference. Towards Reproductive (In)Justice? Mobilities, Technologies, Labourings \& Decisions, , Granada.\filbreak
\item {\bf 2019}: Malgosa, Estel; Mayoral, Marta. {\it Every day in the primary school. Teachers and Education in Gender/Sexual Diversity in Spain.}. ESA Sexuality Research Network. Sociological Explorations of Sexuality in Europe: Bodies, Practices, and Resistance in Troubled Times, , Cracovia (Poland).\filbreak
\item {\bf 2019}: Malgosa, Estel; Mayoral, Marta. {\it (Re)Produciendo sexualidades: Sexo, género y salud sexual y reproductiva en las escuelas de primaria en Cataluña}. IV Coloquio Medical Anthropology Research Center. Sexo, genero y salud., , Tarragona.\filbreak
\item {\bf 2019}: Mantilla, Vanessa. {\it “Ya eres una mujer". Experiencia del cuerpo menstruante en la adolescencia.}. XI AFIN International Conference. Towards Reproductive (In)Justice?: Mobilities, Technologies, Labourings \& Decisions., , Granada.\filbreak
\item {\bf 2019}: Marre, Diana. {\it En busca de los ‘orígenes’: encadenando movilidades íntimas en la reproducción con ‘terceras partes’.}. V Congreso Internacional de Antropología AIBR: Pensar culturas, cambiar mundos, , Madrid (Spain), July 9-12.\filbreak
\item {\bf 2019}: Marre, Diana. {\it Cross-Border Mobilities in Contemporary Spain.}. Territorial Discourses \& Feminist Politics., , Reykjavik, May 22th-24th.\filbreak
\item {\bf 2019}: Marre, Diana. {\it Dilemas en la crianza: reflexiones en torno a la primera infancia.}. Jornada Dilemas en la crianza: reflexiones en torno a la primera infancia., , Madrid: Cruz Roja Española, Fundación Aprender, Grupo n5, Fundación Meniños, Unicef España. April 3..\filbreak
\item {\bf 2019}: Marre, Diana. {\it Reproductive mobilities in contemporary Spain.}. Human Reproduction Study. Group. Winter Event. Critically Mapping Cross-Border Reproduction., , Bristol (UK). February 1st.\filbreak
\item {\bf 2019}: Marre, Diana. {\it Participación en la mesa redonda ¿Es posible seguir manteniendo el anonimato de la donación de gametos en España?}. Simposio Internacional “Donaciones” en Reproducción Asistida: parentesco, biotecnologías y bioeconomía., , Madrid (Spain). October 3-4..\filbreak
\item {\bf 2019}: Marre, Diana. {\it La gestación por sustitución: perspectiva antropológica.}. Ponencia inaugural del curso 2019-2020 del Máster en Biología de la Reproducción y Técnicas de Reproducción Humana Asistida, , Instituto Universitario Dexeus y Departamento de Biología Celular, Fisiología e Inmunología de la Universidad Autónoma de Barcelona. Barcelona (Spain). October 10..\filbreak
\item {\bf 2019}: Marre, Diana. {\it Del turismo médico al 'cross-border care': Una cuestión de definiciones.}. Doctorado en Turismo., , Universidad de Baja California. Tijuana (México). November 25th..\filbreak
\item {\bf 2019}: Marre, Diana. {\it El trauma de escoger el método.}. Doctorado en Turismo., , Universidad de Baja California. Tijuana (México). Noviembre 27..\filbreak
\item {\bf 2019}: Marre, Diana. {\it La adopción de niños y niñas ‘mayores’: de las adopciones “especiales” o la diversidad parental}. 3º Congreso Galego de Adopción e Acollemento. Retos e propostas., , Pontevedra (Spain). 2019, May 31 \& Juny 1..\filbreak
\item {\bf 2019}: Molas, Anna. {\it Gamete donation in Spain: market or gift?}. International symposium Universidad Complutense de Madrid: Reproductive markets: actors, institutions and legal frameworks, , Madrid, Spain..\filbreak
\item {\bf 2019}: Molas, Anna. {\it Beyond the making of altruism: branding and identity in egg donation websites in Spain}. BSA Human Reproduction Study Group conference: Critically Mapping Cross-Border Reproduction, , Bristol, UK.\filbreak
\item {\bf 2019}: Molas, Anna; Perler, Laura. {\it Selecting women, taming bodies? Body ontologies in egg donation practices in Spain}. XI AFIN International Conference: Towards Reproductive (In)Justice: Mobilities, Technologies, Labouring \& Decisions, , Granada, Spain.\filbreak
\item {\bf 2019}: Sanmiquel-Molinero, Laura. {\it Dis/capacitando el género, de/generando la discapacidad: interconexiones entre el hetero/sexismo y el dis/capacitismo}. IV Coloquio MARC: "Sexo, Genero y Salud", 30-31 May 2019, Universitat Rovira i Virgili. 30-05-2019.\filbreak
\item {\bf 2020}: Alvarez, Bruna. {\it Les polítiques de les emocions i la mobilitat reproductiva: receptores d’òvuls britàniques a Barcelona.}. I Congrés Català d’Antropologia (COCA), , Tarragona (España).\filbreak
\item {\bf 2020}: Alvarez, Bruna. {\it Reproducció estratificada i mobilitat en un context d’austeritat i de declivi de la fecunditat: qüestionant la justícia reproductiva. Organizadora de Panel.}. I Congrés Català d’Antropologia (COCA), , Tarragona (España).\filbreak
\item {\bf 2020}: Clemente-Martínez, . {\it Adopcions trasnacionals i els viatges de retorn als orígens}. I Congrès Català d’Antropologia. Institut Català d’Antropologia, , Universitat Rovira i Virgili Tarragona. Tarragona (Spain).\filbreak
\item {\bf 2020}: Clemente-Martínez, Chandra Kala. {\it Families and Strangers: Exclusivity and inclusivity after contacts and reunions in adoption}. Bachelor of Gender Sociocultural Studies. Autonomous University of Barcelona., , Virtual Presentation.\filbreak
\item {\bf 2020}: Clemente-Martínez, Chandra Kala. {\it Searches, contacts, and reuniones in adoptions of Nepalese origin}. Seminar for “Programa de preparación, apoyo y acompañamiento a las familias adoptivas”. MANAIA, Asociación Galega de Adopción e Acollemento., , Virtual Presentation.\filbreak
\item {\bf 2020}: Clemente-Martínez, Chandra Kala. {\it What does involve doing anthropology nowadays?}. Seminar for “Basic Principles of Social Anthropology”. Bachelor of Gender Sociocultural Studies. Autonomous University of Barcelona., , Virtual Presentation.\filbreak
\item {\bf 2020}: Clemente-Martínez, Chandra Kala. {\it La Cerca d’Orígenes a l’Adopción Internacional}. Cicle Cerca d’Orígens. Associació en Defensa del Dret de la Infància a la Família, , Barcelona (Spain).\filbreak
\item {\bf 2020}: Insausti, Santiago Joaquin. {\it Una historia del sexo compensado entre varones en Latinoamerica: los casos de Mexico y Buenos Aires (1940-1990)}. Terceras Jornadas Posdoctorales en Humanidades y Ciencias Sociales, , Ciudad de México, México.\filbreak
\item {\bf 2020}: Insausti, Santiago Joaquin. {\it Ni las mujeres ni los tipos, a mí me gustan los culos. Las masculinidades juveniles frente al nuevo mayatismo mexicano.}. XIV JORNADAS ANDINAS DE LITERATURA LATINOAMERICANA, JALLA MÉXICO 2020: Mundos animales, mundos vegetales, cuerpos y ánimas en la Tierra., , Ciudad de México, México.\filbreak
\item {\bf 2020}: Marre, Diana. {\it El desplazamiento forzado de bebés, niños y niñas en España: dificultades para su reconocimiento y reparación.}. Adopciones e “irregularidades”: historias, contextos, sentidos y desafíos., , Facultad de Psicología, Programa de Doctorado en Psicología y Red nº 170133, Universidad Alberto Hurtado, Santiago de Chile. December 9th..\filbreak
\item {\bf 2020}: Marre, Diana. {\it Diversidad familiar, decisiones y procesos reproductivos en el mundo contemporáneo.}. XXIII Máster de Biología de la Reproducción y Técnicas de Reproducción Humana Asistida., , UAB-Dexeus. October 21st..\filbreak
\item {\bf 2020}: Marre, Diana. {\it Contexto social y de género en relación a las prácticas reproductivas}. 1ª Jornada sobre parto respetado y violencia reproductiva., , Hospital de la Santa Creu y Sant Pau (Barcelona). October 16th.\filbreak
\item {\bf 2020}: Marre, Diana. {\it Diversidad familiar, decisiones y procesos reproductivos en el mundo contemporáneo.}. Máster en Genética Asistencial., , Universidad Autónoma de Barcelona, Unidad de Genética Clínica y Molecular del Hospital Vall d’Hebron de Barcelona y Unidad de Genética del Hospital Sant Pau de Barcelona. May 4th..\filbreak
\item {\bf 2020}: Molas, Anna. {\it ‘Nothing to lose’: Information and risk perception  among egg donors in Spain}. 2020 Annual Social and Political Sciences Graduate Research Symposium (Monash University), , .\filbreak
\item {\bf 2020}: Molas, Anna; Perler, Laura. {\it Selecting women, taming bodies? Body ontologies in egg donation practices in Spain}. EASST/4S 2020 Conference. Locating and Timing Matters: Significance and agency of STS in emerging worlds, , .\filbreak
\item {\bf 2020}: Vilafranca Cartagena, Mireia; Rubinat Arnaldo, Ester; Tort Nassarre, Glòria. {\it Factores condicionantes para la realización de actividad fisica en adultos con Diabetis Mellitus Tipo 2}. I Congreso Virtual Interdisciplinar Iberoamericano de Enfermería y Fisioterapia, , .\filbreak
\item {\bf 2021}: Alvarez et al., Bruna. {\it Changing families. New methods: III.}. LASA, Crisis Global: desigualdades y centralidad en la vida., , Online.\filbreak
\item {\bf 2021}: Alvarez, Bruna. {\it Hablar sin hablar: maestros y sexualidad en las escuelas de primaria}. 7 International Anthropological Conference AIBR: Humanidad: unidad y diversidad., , Vila-Real (Portugal).\filbreak
\item {\bf 2021}: Alvarez, Bruna; Malgosa, Estel. {\it Children and sexuality. Panel organizer.}. 7 International Anthropological Conference AIBR: Humanidad: unidad y diversidad., , Vila-Real (Portugal).\filbreak
\item {\bf 2021}: Casanova Molina, Gloria. {\it Cuerpos ficticios, identidades falseadas: la construcción socio-médica de las intersexualidades.}. Congreso Política y Diferencia (UAB-USP): Nuevas perspectivas antropológicas para el Siglo XXI, , Online.\filbreak
\item {\bf 2021}: Clemente-Martínez, Chandra Kala. {\it La investigación etnográfica multisituada desde la antropología social}. Laboratorio de Metodologías Innovadoras en Investigación Social. Instituto de Investigaciones en Ciencias del Comportamiento, , Virtual Presentation. La Paz (Bolivia).\filbreak
\item {\bf 2021}: Clemente-Martínez, Chandra Kala. {\it La construcción social de la identidad desde la antropología}. 4º Congreso MANAIA: En Primera Persona, , Pontevedra (Spain).\filbreak
\item {\bf 2021}: Clemente-Martínez, Chandra Kala. {\it Going Back to the Origins: Old and New Travels in Transnational Adoptions}. 12th European Congress on Tropical Medicine and International Health, , Virtual Presentation. Bergen (Norway).\filbreak
\item {\bf 2021}: Clemente-Martínez, Chandra Kala. {\it Birth Families’ Experience in Searching and Reuniting}. Eighth Biennial Virtual Conference: Adoption, Kinship, Culture: Engaging the Past, Imagining the Future. Alliance for the Study of Adoption and Culture, , Virtual Presentation.\filbreak
\item {\bf 2021}: Desy, Alexandra. {\it L’AMP transfrontalières avec don de gamètes et la question des origines : le voyage reproductif des français.e.s en Espagne.}. Le «droit aux origines» :
du domaine de l’adoption à celui de la PMA., , Université de Lausanne (Suiza).\filbreak
\item {\bf 2021}: García-Santesmases, Andrea; Sanmiquel-Molinero, Laura. {\it "Al final te acostumbras a todo: desvelando y problematizando la producción del cuerpo-sujeto discapacitado}. VII Conferencia AIBR, , Vila-Real (Portugal).\filbreak
\item {\bf 2021}: Insausti, Santiago Joaquin. {\it Cruces e intersecciones entre sexualidades, género y clases.}. V Jornadas de Trabajo de la Red de Estudios sobre Represión y Violencia Política, , Buenos Aires, Argentina.\filbreak
\item {\bf 2021}: Insausti, Santiago Joaquin. {\it Veinte años de teteras entre México y Buenos Aires (1997-2017)}. Coloquio Internacional: ?Memoralia de aceras olvidadas (ligues, yiros y cruisings en Latinoamérica)., , Lleida, España.\filbreak
\item {\bf 2021}: Insausti, Santiago Joaquín. {\it Cartografías de las redes queer trasatlánticas en la década de los 70}. Tercer Congreso Internacional Género y sexualidad en las culturas hispánicas, , Lleida, España.\filbreak
\item {\bf 2021}: Malgosa, Estel. {\it Placeres, embarazos y cuerpos: Narrativas infantiles sobre sexualidad.}. VII Congreso Internacional de Antropología AIBR. Humanidad: igualdad y diversidad., , Vila Real (Portugal).\filbreak
\item {\bf 2021}: Malgosa, estel. {\it SexAFIN-Barcelona: Retos y limitaciones de las metodologías participativas para la investigación en "sensitive topics".}. International Congress of the Latin American Studies Association. Desigualdades y centralidad de la vida., , Online.\filbreak
\item {\bf 2021}: Mantilla, Vanessa. {\it El tema oculto en la adolescencia: la menstruación.}. 7º Congreso Internacional de Antropología AIBR, Humanidad: unidad y diversidad., , Portugal.\filbreak
\item {\bf 2021}: Mantilla, Vanessa; Desy, Alexandra. {\it Endométriose et mobilité reproductive. Une analyse comparée entre la France et l’Espagne.}. Endométriose et Inégalités. Expériences, expertises et problème public., , Paris (France) - Maison des Sciences de L'Homme..\filbreak
\item {\bf 2021}: Marre, Diana. {\it Reproducción, contracepción y sexualidad}. , , .\filbreak
\item {\bf 2021}: Marre, Diana. {\it La difícil conciliación de la vida laboral y familiar: entre el control de la natalidad, la preservación de la fertilidad y la externalización de la reproducción.}. Seminario Internacional Mujeres, equidad y derechos., , Universidad Autónoma de Ciudad Juárez y Asociación de Académicas del Estado de Chihuahua. Ciudad Juárez (México), February 24..\filbreak
\item {\bf 2021}: Molas, Anna; Whittaker, Andrea. {\it A world without surrogates and donors: sociotechnical fantasies and futures}. Australian Anthropological Society Conference, , Online.\filbreak
\item {\bf 2021}: Roca, Judit; Reguant, Mercedes; Tort-Nassarre, Glòria. {\it Un dispositivo pedagógico para integrar aspectos de seguridad del paciente en simulaciones clínicas}. XIX Jornadas de Redes de Investigación en docencia Universitaria-REDES 2021, , .\filbreak
\item {\bf 2021}: Sanmiquel-Molinero, Laura. {\it Adjustment to disability meets dis/ability studies: bridging two historically confronted approaches to understand disability sex and relationships}. How can Feminist Dis/ability Studies help us unpack sexuality \& disability?, 25-25 November 2021, iHuman Institute.\filbreak
\item {\bf 2021}: Sanmiquel-Molinero, Laura. {\it Una mirada crítica a "Estigma" de Erving Goffman desde los estudios de la Discapacidad}. Ciclo Permanente de Conferencias del Posgrado en Psicología Social, , Universidad Autónoma Metropolitana Unidad Iztapalapa.\filbreak
\item {\bf 2021}: Sanmiquel-Molinero, Laura. {\it Adjustment to Disability: Exploring What It Means, How It Feels and How It Is Done through Two Qualitative Techniques}. 9th ESPA European Conference, 06-05 July 2021, Online.\filbreak
\item {\bf 2021}: Sanmiquel-Molinero, Laura; García-Santesmases, Andrea. {\it Adjusting to an/other sexuality after an SCI: Negotiating ableism and heterosexism}. ESA Conference, 31-09 March 2021, Barcelona.\filbreak
\item {\bf 2021}: Sanmiquel-Molinero, Laura; García-Santesmases, Andrea. {\it “They Were Shaving My Head and It Was Even More Upsetting than the SCI Itself”: Liminality and De/sexualization of the Disabled Body}. Disabilities/Arts/ Sexualities Conference, 11-03 December 2021, Dublin City University.\filbreak
\item {\bf 2022}: Alvarez, Bruna. {\it “Parla-ho amb la teva mare!”: mestres, sexualitat i infància a les escoles de primària.}. COCA: II Congrés Català d’Antropologia, , Girona (España).\filbreak
\item {\bf 2022}: Alvarez, Bruna; Desy, Alexandra. {\it AFIN-ART Support Service: anthropological practice in assisted reproduction and reproductive mobilities.}. Mobilising Methods in Medical Anthropology. Glocal Health Methods Festival., , Royal Anthropological Institute of Great Britain and Ireland.\filbreak
\item {\bf 2022}: Alvarez, Bruna; Malgosa, Estel. {\it Simposi 34. Sexualitat i infància. La sexualitat governada. Panel Organizer.}. COCA: II Congrés Català d’Antropologia, , .\filbreak
\item {\bf 2022}: Desy, Alexandra. {\it French Reproductive Travel to Spain and the Impact of the Covid-19 Pandemic on Cross-border Reproductive Mobility}. Fulda International Autumn School 2022 “Mobilities and Human Rights”, , Fulda University of Applied Sciences, Germany.\filbreak
\item {\bf 2022}: Desy, Alexandra. {\it Las fronteras a prueba en la reproducción transfronteriza}. I Congreso Internacional Virtual AFIN sobre salud reproductive de la mujer: "Mujeres: entre el Mal-estar (in)visible y el Bien-estar (im)posible", , Barcelona, Spain.\filbreak
\item {\bf 2022}: Desy, Alexandra. {\it “Late” Motherhood and Reproductive Exclusion. Cross-border Reproductive Journeys of French Women and Couples to Barcelona}. BSA Medical Sociology Conference 2022, , British Sociological Association, UK.\filbreak
\item {\bf 2022}: Desy, Alexandra. {\it “Late” Motherhood and Reproductive Exclusion. Cross-border Reproductive Journeys of French Women and Couples to Barcelona}. 8th Congress of the Portuguese Anthropological Association, , Universidade de Évora, Portugal.\filbreak
\item {\bf 2022}: Desy, Alexandra. {\it Maternidades "tardías" y movilidades reproductivas}. Seminarios AFIN 2022, , Barcelona, Spain.\filbreak
\item {\bf 2022}: Desy, Alexandra. {\it Movilidades reproductivas transnacionales: gametos donados y gametos propios / Legislaciones de los países de origen de estas movilidades}. I Congreso REPROMOB, , Barcelona, España.\filbreak
\item {\bf 2022}: Desy, Alexandra; Clemente-Martínez, Chandra Kala. {\it Reproductive imaginaries in Gamete and Embryo Donations and Adoptions}. La parenté par les enfants. Interroger la descendance. Université d’automne du programme ANR Origines., , Centre Norbert Elias (CNRS), France.\filbreak
\item {\bf 2022}: Desy, Alexandra; Gaggiotti Marre, Sofía. {\it El acompañamiento médico y antropológico en recorridos de preservación de la fertilidad en niño/as y adolescentes en tratamiento oncológico}. XV Jornadas Internacionales de Psico-Oncología Infanto-Juvenil, , Fundación Natalistas Dafne Flexer, Argentina.\filbreak
\item {\bf 2022}: Marre, Diana. {\it Hablemos de vulnerabilidad, mujeres y organizaciones}. Laboratorio de Innovación Social Vulnerabilidad Organizacional. Universidad de Baja California., , Tijuana (México).\filbreak
\item {\bf 2022}: Sanmiquel-Molinero, . {\it Problematising Inclusivity through Narrative-Ethnographic Approaches}. Considering Inclusive Research Methodologies, 08-06 August 2022, iHuman Institute.\filbreak
\item {\bf 2022}: Sanmiquel-Molinero, ; Prous-Climent, Elena. {\it Superarse a una misma, inspirar al Otro: Una aproximación (auto)etnográfica a la figura de la "superlisiada"}. I Congreso de Antropología Feminista, 08-06 October 2022, Universidad del País Vasco.\filbreak
\item {\bf 2022}: Sanmiquel-Molinero, ; Pujol-Tarrés, Joan; Montenegro, Marisela. {\it Constitución subjetiva de cuerpos disidentes: infancia trans e infancia discapacitada}. I SexAFIN International Virtual Conference (Re)thinking comprehensive sexuality education from a child-centered and gendered perspective, 13-15 July 2022, Universitat Autònoma de Barcelona.\filbreak
\item {\bf 2022}: Sanmiquel-Molinero, Laura; Pujol Tarrés, Joan; Montenegro, Marisela. {\it Explorando las trayectorias reproductivas de las personas con discapacidad desde el anticapacitismo}. Primer Congreso ReproMob, 17-19 February 2022, Universitat Autònoma de Barcelona.\filbreak
\item {\bf 2023}: Cerezuela González, Ana  and Carolina Remorini. {\it The empathic witness: reflections from the field on the analytical potential of emotions in hospital ethnography}. MAYS  Workshop “The Hospital in Transit”: “Reflexivity: Hospital and the Ethnographers”., , .\filbreak
\item {\bf 2023}: Desy, Alexandra. {\it An Anthropological Approach to Support: innovations for patients and reproductive health professionals to fill the gaps in the medical journeys}. IIT Department of Liberal Arts' Seminar, , Hyderabad, India.\filbreak
\item {\bf 2023}: Desy, Alexandra. {\it French Reproductive Exclusion: exploring Surveillance Mechanisms in Medically Assisted Reproductive Health}. 19th IUAES-WAU World Anthropology Congress 2023, , New Delhi, India.\filbreak
\item {\bf 2023}: Desy, Alexandra. {\it Exclusiones e Inclusiones Reproductivas en Europa, el Caso de las Parejas y Mujeres Francesas que Viajan a España para acceder a Tratamientos de Reproducción Médicamente Asistida}. Repensando las Tecnologías Reproductivas, , Bioeticar Asociación Civil, Argentina.\filbreak
\item {\bf 2023}: Desy, Alexandra. {\it Francia no es País para Madres Solas, Lesbianas o ‘Viejas’}. Género - Exclusiones, Visibilidades, Representaciones, , Escuela Nacional de Antropología e Historia, México.\filbreak
\item {\bf 2023}: Desy, Alexandra. {\it Cuestionar lo Incuestionable: la importancia de la Voz y del Diálogo en el Acompañamiento Antropológico en Reproducción Asistida}. 9º Congreso Internacional de Antropología - AIBR, , Ciudad de México, México.\filbreak
\item {\bf 2023}: Desy, Alexandra. {\it L'Accompagnement Anthropologique dans le cadre des Voyages Reproductifs}. Les Mobilités reproductives : regards croisés, , Paris, France.\filbreak
\item {\bf 2023}: Desy, Alexandra. {\it Building Bridges between Research and Applied Anthropology}. 83rd Annual Meeting of the Society for Applied Anthropology, , Oklahoma City, USA.\filbreak
\item {\bf 2023}: Desy, Alexandra. {\it Negotiating with French reproductive surveillance: French women and couples travelling to Barcelona to access reproductive treatments}. II Congreso REPROMOB, , Barcelona, Spain.\filbreak
\item {\bf 2023}: Desy, Alexandra. {\it Reproductive imaginaries in Gamete and Embryo Donations and Adoptions.}. 2023 conference of the Finnish Anthropological Society: "Relations and Beyond", , Rovaniemi, Finland.\filbreak
\item {\bf 2023}: Desy, Alexandra; Marre, Diana. {\it Movilidades Reproductivas Transfronterizas Franco-Españolas}. Primer Diálogo Social Sobre Biotecnologías Reproductivas, , Montevideo, Uruguay.\filbreak
\item {\bf 2023}: Desy, Alexandra; Marre, Diana. {\it Building Links between Research and Applied Anthropology}. Changing Repro-Ethnographies: Mobility and Translation in the Anthropology of Reproduction, , Boloña, Italia.\filbreak
\item {\bf 2023}: Marre, Diana. {\it Indagando el ámbito sanitario: la práctica, la investigación y la formación de profesionales desde una mirada interdisciplinar}. Universidad Nacional Arturo Jauretche, , .\filbreak
\item {\bf 2023}: Marre, Diana. {\it Derecho à origen}. Instituto de Psicología, Centro de Educaçáo e Humanidades. Universidade do Estado do Rio de Janeiro, , Rio de Janeiro.\filbreak
\item {\bf 2023}: Marre, Diana. {\it La dimensión social de la salud reproductiva: una tarea pendiente / A dimensão social da saúde reprodutiva: uma tarefa pendente}. Instituto de Psicología, Centro de Educaçáo e Humanidades. Universidade do Estado do Rio de Janeiro, , Rio de Janeiro.\filbreak
\item {\bf 2023}: Marre, Diana. {\it El anonimato en la donación de gametos}. , , Barcelona.\filbreak
\item {\bf 2023}: Martone, Paula. {\it Hospital fieldwork as a social experimental setting:  
Insights from a paediatric and maternal hospital in Barcelona}. Hospital Ethnography Conference, MAYS 2023 (Medical Anthropology Young Scholars Association), , Online.\filbreak
\item {\bf 2023}: Martone, Paula; Molas, Anna. {\it Negotiating 'viability' of preterm infants on the ground: the experiences of parents and healthcare workers in Spain}. Technodeath Conference, , .\filbreak
\item {\bf 2023}: Remorini, Carolina and Anna Molas Closas. {\it Más allá del cuerpo individual en los discursos sobre la relación entre ambiente y salud perinatal}. Jornada de Salud Medioambiental. Hospital San Joan de Déu, , Barcelona.\filbreak
\item {\bf 2023}: Remorini, Carolina y Anna Molas Closas. {\it Contribuciones de la antropología al estudio de la salud materno-infantil en contextos hospitalarios}. 1ra Jornada de Investigación RICORS-SAMID; Congreso de Nacional de Neonatología y Medicina Perinatal, , Santiago de Compostela.\filbreak
\item {\bf 2023}: Remorini, Carolina. {\it 3.5.	Learning and mutual raising: children's informal learning in rural settings (Argentina).}. Jean Piaget Society Conference, , Madrid.\filbreak
\item {\bf 2023}: Remorini, Carolina. {\it Qué tiene que ver la cultura con el desarrollo infantil? Diálogos para una mayor inclusión de la Antropología en campo del (neuro)desarrollo infantil}. I Jornadas de actualización pediátrica “Hacia una mirada integral en el desarrollo infantil, , La Plata, Argentina.\filbreak
\item {\bf 2023}: Sanmiquel-Molinero, Laura. {\it Desgranando el dis/capacitismo: hacia unas políticas de cuidado críticas e interseccionales}. IX Congreso de la Red Española de Política Social, 25-27 October 2023, Palma de Mallorca.\filbreak
\item {\bf 2023}: Sanmiquel-Molinero, Laura. {\it Introducción a los estudios críticos de la discapacidad}. , 09-10 December 2023, Universidad Autónoma de Barcelona, Bellaterra, España.\filbreak
\item {\bf 2023}: Sanmiquel-Molinero, Laura; Montenegro, Marisela; Gutierrez Monclús, Pamela; Pujol Tarrés, Joan. {\it Perspectivas Latinoamericanas sobre crianza y discapacidad: Vulnerabilidad, riesgo e inclusión social}. XIII Biennial ISCHP Conference, 26-29 July 2023, Rancagua, Chile.\filbreak
\item {\bf 2023}: Sanmiquel-Molinero, Laura; Montenegro, Marisela; Pujol Tarrés, Joan. {\it Discapacidad y parentalidad en América Latina: explorando la tensión entre la vulnerabilidad, el riesgo y los derechos}. Segundo Congreso ReproMob, 27-29 March 2023, Universitat Autònoma de Barcelona.\filbreak
\item {\bf 2023}: Sanmiquel-Molinero, Laura; Pujol-Tarrés, Joan; Montenegro-Martínez, Marisela. {\it Liminalidad Reproductiva La confluencia del riesgo y la vulnerabilidad en las trayectorias reproductivas de personas con discapacidad}. Congreso Internacional hacia la materialización de la ficción: agencias, resistencias y revueltas en la contemporaneidad, 26-28 July 2023, San José Costa Rica.\filbreak
\item {\bf 2024}: Andreica-Gheorghe, Zenaida-Maria. {\it Imágenes en temas sensibles: el uso de AI como herramienta visual en el estudio de las practicas digitales online}. 10º Congreso internacional de Antropología AIBR, 09-07 December 2024, Madrid.\filbreak
\item {\bf 2024}: Andreica-Gheorghe, Zenaida-Maria. {\it Children’s perspectives on pornography and other online practices in Spain}. 18th EASA Biennial Conference, , Barcelona.\filbreak
\item {\bf 2024}: Desy, Alexandra. {\it Bioética y Biopolítica en diálogo internacional}. Seminario UBA, , Facultad de Psicología - Universidad de Buenos Aires, Argentina.\filbreak
\item {\bf 2024}: Desy, Alexandra. {\it Between borders and rights: French women's cross-border reproductive
journeys to Spain to become single mothers}. I Colóquio Rede Anthera: Desafios em Governança Reprodutiva. Rede Internacional de Pesquisa sobre Família e Parentesco
(Rede Anthera), , Porto Alegre, Brasil.\filbreak
\item {\bf 2024}: Desy, Alexandra. {\it Beyond borders: the power of online support networks in French cross-border fertility journeys}. Tercer Congreso Internacional ReproMob - Entre embates conservadores y conquistas de derechos: vulnerabilidades, inequidades y justicia reproductiva, , Buenos Aires, Argentina.\filbreak
\item {\bf 2024}: Desy, Alexandra. {\it "Sentía que estaba haciendo algo ilegal": El itinerario de las mujeres francesas hacia la maternidad en solitario}. Seminarios AFIN 2023-24, , Barcelona, Spain.\filbreak
\item {\bf 2024}: Desy, Alexandra. {\it The therapeutic potential of anthropologically framed counselling in fertility journeys}. RAI2024: Anthropology and Education, , London, UK.\filbreak
\item {\bf 2024}: Desy, Alexandra. {\it Navegar por los imperativos de la procreación confrontando las contradicciones entre los discursos conservadores y neoliberales y las expectativas de fecundidad en Francia}. Tercer Congreso Internacional ReproMob - Entre embates conservadores y conquistas de derechos: vulnerabilidades, inequidades y justicia reproductiva, , Buenos Aires, Argentina.\filbreak
\item {\bf 2024}: Desy, Alexandra; Marre, Diana; Remorini, Carolina. {\it The invisible work involved in early gestational loss in Spain}. 17th Equality, Diversity, \& Inclusion conference, , Sevilla, Spain.\filbreak
\item {\bf 2024}: Martone, Paula; Molas, Anna. {\it Liminal Babies, Liminal Parents: Personhood Constructions at the NICU and its Implications for the Artificial Placenta}. EASA Conference, , Universitat de Barcelona.\filbreak
\item {\bf 2024}: Salvo Agoglia, Irene. {\it Exposición SANKOFA}. WORKSHOP INFANCIA\_C10  CHALLENGES AND EXPECTATIONS IN CHILD RESEARCH, , Universidad Autónoma de Madrid, España.\filbreak
\item {\bf 2024}: Salvo Agoglia, Irene. {\it Producciones creativas y participativas en el campo de la adopción: Algunas reflexiones sobre sus aportes, límites y proyecciones}. WORKSHOP INFANCIA\_C10  CHALLENGES AND EXPECTATIONS IN CHILD RESEARCH, , Universidad Autónoma de Madrid.\filbreak
\item {\bf 2024}: Salvo Agoglia, Irene. {\it Imaginando la adopción abierta en Chile: perspectivas y tensiones de profesionales y familias adoptivas}. 10th AIBR International Conference of Anthropology, , Madrid.\filbreak
\item {\bf 2024}: Sanmiquel-Molinero, ; Montenegro, Marisela; Pujol Tarrés, Joan. {\it Reflexiones metodológicas sobre el estudio de la discapacidad, la sexualidad y la reproducción}. Congreso Internacional sobre Estudios de Diversidad Sexual y de Género CIEDSI, 17-19 July 2024, Universitat de Girona.\filbreak
\item {\bf 2024}: Sanmiquel-Molinero, Laura; García-Santesmases Fernández, Andrea. {\it Hacia una ética feminista y anticapacitista de los cuidados}. Jornada sobre Salud Pública Feminista: aportaciones desde las éticas y las epistemologías, 12-06 December 2024, Universitat Autònoma de Barcelona.\filbreak
\item {\bf 2024}: Sanmiquel-Molinero, Laura; Montenegro, Marisela; Pujol Tarrés, Joan. {\it Espacios sanitarios, espacios “de riesgo”: explorando las experiencias de las mujeres discapacitadas que son o buscan ser madres}. Tercer Congreso ReproMob, 08-04 October 2024, Universitat Autònoma de Barcelona.\filbreak
\item {\bf 2024}: Sanmiquel-Molinero, Laura; Montenegro, Marisela; Pujol Tarrés, Joan. {\it Maternidad discapacitada y riesgos para la descendencia: un análisis desde los Estudios Críticos de la discapacidad}. 18th EASA Biennial Conference, 18-26 July 2024, Online.\filbreak
\item {\bf 2024}: Sanmiquel-Molinero, Laura; Montenegro, Marisela; Pujol Tarrés, Joan. {\it Explorando las trayectorias reproductivas de las personas discapacitadas desde una mirada feminista y anticapacitista.}. II Congreso Internacional de Antropología Feminista, 01-07 March 2024, Universidad de Granada.\filbreak
\item {\bf 2024}: Sanmiquel-Molinero, Laura; Pujol Tarrés, Joan; Montenegro, Marisela. {\it El enwheelment como orientación liminal al espacio-tiempo: reflexiones en torno a las (in)movilidades reproductivas}. X Congreso AIBR, 09-07 December 2024, Universidad Complutense de Madrid.\filbreak
\end{enumerate} 
% }}}
