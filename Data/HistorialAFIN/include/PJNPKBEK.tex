% d3 {{{
\begin{enumerate}
\item {\bf 2019}. Mantilla Salazar, Vanessa. {\it “Ya eres una mujer” La experiencia del cuerpo menstruante en la adolescencia}. , , Autònoma de Barcelona. Directed by: Diana Marre, directora.\filbreak
\item {\bf 2019}. McIntyre, Lynne. {\it Ho vull enterrar: the use of ritual in perinatal loss among women in Catalonia}. , , Autònoma de Barcelona. Directed by: .\filbreak
\item {\bf 2020}. García Julve, Silvia. {\it New Social Movements based on the canons of beauty}. , , Autònoma de Barcelona. Directed by: Diana Marre, directora.\filbreak
\item {\bf 2020}. Pinzón Sanabria, Diana Paola. {\it Educación afectivo-sexual de adolescentes en Manresa, Cataluña: sus percepciones y expectativas}. , , Autònoma de Barcelona. Directed by: Diana Marre, directora.\filbreak
\item {\bf 2021}. Campos Lichtsztejn, Mariana. {\it Etnocentrismo y políticas reproductivas: Un análisis de la violencia obstétrica desde una perspectiva antropológica}. Máster oficial en Antropología Social: Investigación Avanzada e Intervención Social, Department of Antropologia Social i Cultural, Universitat Autònoma de Barcelona. Directed by: Diana Marre.\filbreak
\item {\bf 2021}. Desy, Alexandra. {\it Los viajes reproductivos transfronterizos. Mujeres y parejas francesas en Barcelona}. , , De Barcelona. Directed by: Diana Marre, Silvia De Zordo, codirectoras.\filbreak
\item {\bf 2021}. Díaz Alday, Natalia. {\it El “Interés Superior de la Niñez”: significados construidos por profesionales del “Programa Mi Abogado}. , , Universidad Alberto Hurtado. Directed by: Calificación máxima: 7 (en Chile).\filbreak
\item {\bf 2021}. Torra Merín, Martí. {\it “Això no es pot dir”. La regulació de la sexualitat a través de la comunicació en escoles d’educació primària de Catalunya}. , , Autònoma de Barcelona. Directed by: Bruna Alvarez, Diana Marre, codirectoras.\filbreak
\item {\bf 2022}. Andreica-Gheorghe, Zenaida. {\it Send nudes: Whatsapp e Instagram, un solo panóptico del amor}. , , Universitat Autònoma de Barcelona. Directed by: Bruna Alvarez y Monstserrat Juneda, directoras.\filbreak
\item {\bf 2022}. Astudillo, Matías. {\it Entre la familia de origen y la familia de acogida externa:  Relaciones, afectos y contactos en el proceso de acogimiento familiar}. , , Universidad Diego Portales. Directed by: .\filbreak
\item {\bf 2022}. Hanafi Alcolea, Dunia. {\it Evaluación cualitativa de impacto en proyectos sociales: ejemplo de SexAFIN.}. , , Universitat Autònoma de Barcelona. Directed by: Bruna Alvarez, directora.\filbreak
\item {\bf 2023}. Batuman, Idil. {\it What if the dormant snake awakens...? An ethnographic study about kundalini yoga practices in Spain and Turkey}. , , Universitat Autònoma de Barcelona. Directed by: Bruna Alvarez, directora.\filbreak
\item {\bf 2023}. Farias, Dafne. {\it Adopción y relaciones fraternas en Chile: Experiencias, desafíos y proyecciones para garantizar el principio de inseparabilidad de los hermanos(as) en la adopción}. , , Universidad Alberto Hurtado. Directed by: .\filbreak
\item {\bf 2023}. Fernández, Raquel. {\it Menstruaciones periféricas. Aproximación etnográfica a la construcción de la identidad menstruante en cuerpos no hegemónicos}. , , Autònoma de Barcelona. Directed by: Diana Marre, directora.\filbreak
\item {\bf 2023}. Navarro Bueno, Jesica. {\it Mapeando el estiga corporal puta. Un estudio de caso de trabajadoras sexuales en Barcelona.}. , , Universitat Autònoma de Barcelona. Directed by: Bruna Alvarez y Livia Moterle, directoras.\filbreak
\item {\bf 2023}. Salas, Pamela. {\it Desafíos actuales en la clínica de la adopción en Chile: la construcción de continuidades psicoterapéuticas en niños y niñas adoptados menores de tres años de edad}. , , Universidad Alberto Hurtado. Directed by: .\filbreak
\item {\bf 2023}. Soto, Martha. {\it Hablando de las rupturas adoptivas en Chile: narrativas de profesionales especializados en adopción}. , , Universidad Alberto Hurtado. Directed by: .\filbreak
\item {\bf 2023}. Urzúa, Bárbara. {\it La historia no es destino”. Análisis de un caso de acogimiento familiar de larga estancia como alternativa de restitución del derecho a vivir en familia}. , , Universidad Alberto Hurtado. Directed by: .\filbreak
\item {\bf 2024}. Cerezuela, Ana. {\it Deseos reproductivos, decisiones médicas: la construcción social del embarazo “de riesgo” en los espacios de seguimiento obstétrico}. , , UNED. Directed by: Dir.: Diana Marre, Nancy Konvalinka.\filbreak
\end{enumerate} 
% }}}
